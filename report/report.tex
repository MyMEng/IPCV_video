% include a brief discussion about how the limitation of leaf instances affects the performance of the decision tree algorithm.

\documentclass[12pt,a4paper,twocolumn]{article}
\usepackage{times} % times font
\usepackage{mathptmx} % times font in maths
% \usepackage{fullpage}
\usepackage[top=1in, bottom=1in, left=1in, right=1in]{geometry}
\usepackage{multirow} %in tables
\usepackage{caption} % in tables
\pagenumbering{gobble}
\newcommand{\HRule}{\rule{\linewidth}{0.5mm}}
\usepackage{lipsum}

\begin{document}

\twocolumn[
\begin{@twocolumnfalse}
\begin{center}
	\begin{large}
	{\HRule \\[0.2cm]}
	\textsc{Assignment Motion Estimation and Gesture Recognition}
	{\HRule \\[0.3cm]}
	\end{large}

	\begin{minipage}{ 0.45\textwidth }
		\begin{flushleft}
			Kacper \textbf{Sokol}--- \texttt{ks1591} --- 3GGK1\\
			Maciej \textbf{Kumorek}--- \texttt{mk0934}--- 3G403\
		\end{flushleft}
	\end{minipage}
	\begin{minipage}{ 0.45\textwidth }
		\begin{flushright}
			{COMS30121 $|$ Assignment 03\\
			Image Processing \& Computer Vision\\[0.3cm]}
		\end{flushright}
	\end{minipage}
\end{center}
\end{@twocolumnfalse}
] % \lipsum[1]~\\[0.4cm]
\section*{Introduction}
Our goal in this assignment was to implement Lucas-Kanade (LK) based motion detection program. This approached is based on horizontal, vertical and time based derivatives of a sequence of images. It is simplistic solution for motion detection task with some restrictions but ease of implementation and overall performance seems reasonable arguments to use this approach.

\section*{Implementation and approach}
\lipsum[1]

\section*{Program features}
We have implemented a few different features that can be turned on by specifying the following arguments for the program:
\begin{itemize}
\item \texttt{--threshold=\textit{\#}}--- overwrite implemented thresholding function with fixed value,
\item \texttt{--regions}--- enables user defined regions for motion tracking,
\item \texttt{--showall}--- this switch enables grid of motion tracking points with attached unit vectors showing movement direction,
\item \texttt{--regionSize=\textit{\#}}--- defines the size of region for derivative calculation.
\end{itemize}

\section*{Deliverables}
\begin{itemize}
\item Literature survey of currently used semi-supervised learning algorithms.
\item Basic analysis of the algorithms.
\item Applications of such algorithms.
\end{itemize}

\end{document}
